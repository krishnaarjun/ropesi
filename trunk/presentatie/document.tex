\documentclass{beamer}
\usetheme{cupertino}
\usepackage{amsmath} % formule matematice
\usepackage[noend]{algorithmic}
\usepackage{algorithm}

\setbeamertemplate{footline}[frame number]
%\usepackage{beamerthemesplit}
%\usetheme{classic} %\usetheme{Boadilla} %\usetheme{shadow} 
\definecolor{gray}{rgb}{0.6,0.6,0.6}
\newcommand{\fade}[1]{\textcolor{gray}{#1}}
\title{Rock!Paper!Scissors!}
\author{Silvia Laura Pintea \& Nimrod Raiman}
\date{\fade{\{6109969, 0336696\}}}
\begin{document} 
\frame{\titlepage}
\section[Outline]{}
\frame{\tableofcontents}
\definecolor{orange}{rgb}{1,0.8,0.4}
\newcommand{\todo}[1]{\textcolor{orange}{\emph{#1}}}
\fontfamily{cmbr}\selectfont %ccr%cmbr%cmss
\newcommand{\high}[1]{\textbf{\underline{#1}}}

\section{Task Description}
\frame{
  \frametitle{Task Description}
	Learn Nao to play rock, paper scissors.
	\begin{itemize}
	\item learn a model for hand gestures of rock, paper and scissors 
	\item extract hand from web cam stream 
	\item Classify extracted hand
	\item Make Nao make rock!paper!scissors! moves and play the game
	\end{itemize}
	\vspace*{10px}
}
 
%new approach section_______________________________________________________
\section{Approach}
\frame{
  \frametitle{Approach}
	\begin{enumerate}
	\item Extract hands from web cam stream\\
		\begin{itemize}
		\item detect face
		\item get probability histogram
		\item backproject probability histogram on frame
		\item extract hand segment
	    \end{itemize}	
    \vspace*{9px}
	\item Extract features from hand gestures\\
		\begin{itemize}
		\item PCA
		\item Gabor filters
	    \end{itemize}	
    \vspace*{9px}
	\item Learn classifier for rock, paper and scissors gestures\\
		\begin{itemize}
		\item SVN
		\item KNN
        \end{itemize}
    \vspace*{9px}
	\item Get this running on Nao
	\end{enumerate}
}

\subsection{Extracting hand from the frames}
\frame{
  \frametitle{Extracting hand from the frames}
  Naive approach (as shown in a lot of papers)
	\begin{itemize}
	\item determin hue and saturation values for skin color
	\item threshold the image with hue and saturation values
	\item erode \& dilate
	\item find hand ares\\
	\item this approach is very sensitive for background noise. Don't do this!
	waste of time!!
    \end{itemize}
}

\frame{
  \frametitle{Extracting hand from the frames}
  Smart solution:
	\begin{itemize}
	\item determin skin color histogram
		\begin{itemize}
		\item Detect face
		\item build histogram of inside of face
	    \end{itemize}
	\item backproject skin color histogram on hole frame
	\item erode\&dilate to loose noise and fill up holes
	\item extract area of hand
	\item more sophisticated erotion\&dilation on hand area
		\begin{itemize}
		\item keep only hand and get rid of background
		\item resize to 70x70
        \end{itemize}
	\item skin color is adapted to user
	\item to some extent scale invariant
    \end{itemize}
}

\subsection{Extract features of hand images}
\frame{
  \frametitle{Extract features of hand images}
	\begin{itemize}
	\item PCA
		\begin{itemize}
		\item Dimensionality reduction
	    \end{itemize}
	\item Gabor filters
		\begin{itemize}
		\item Some mambo jambo formulas ;)
	    \end{itemize}
	\item PCA on concatianted Gabor filters
    \end{itemize}
}

\subsection{Learn classifier for rock, paper and scissors gestures}
\frame{
  \frametitle{Learn classifier for rock, paper and scissors gestures}
	\begin{itemize}
	\item SVM
	\item KNN
    \end{itemize}
    on raw images, PCA and gabor filter response
}

\subsection{Get this running on Nao}
\frame{
  \frametitle{Get this running on Nao}
	\begin{itemize}
	\item Nao is sweet (naknak)
	\item 500MhZ processor
	\item 3 fingers that move simultaniosly
	\item broke his hip 2 times in 6 month
	\item costs 17.500 dollar (16.800k if you buy 5)
	\item aditional 3 year maintanance 4.800 dollar
    \end{itemize}
} 

%Asumptions section___________________________________________________________________________________
\section{Assumptions \& Experimental Set-up}
\subsection{Assumptions}
\frame{
  \frametitle{Assumptions}
	experimental set-up. Do we have this? :)
}

\subsection{Experimental Set-up}
\frame{
  \frametitle{Experimental Set-up}
	experimental set-up
}

%Evaluation section___________________________________________________________________________________
\section{Evaluation}
\frame{	
  \frametitle{Evaluation}
	\begin{itemize}
	\item Table with results for SVN and KNN for raw images, PCA and Gabor
		\begin{itemize}
		\item perfect learning set
		\item extracted learning set
		\end{itemize}	
	\end{itemize}	
}
%conclusions section____________________________________________________________________________________
\section{Conclusions}
\frame{
  \frametitle{Conclusions}
	\begin{itemize}
	\item SVM can handle only very simple cases where only one orientation per
	gesture is allowed \vspace*{8px}
	\item PCA was not strong enough to extract features when all orientations for
	gestures are allowed \vspace*{8px}
	\item KNN works best. Even on raw images
	\vspace*{8px}
	\item when not perfect images are used (those extracted by our method) results
	drop, but when classifier is trained on less perfect images results are
	better\ldots????
	\end{itemize}
}

\end{document}
